\documentclass[12pt]{article}

\usepackage{fullpage}
\usepackage{multicol,multirow}
\usepackage{tabularx}
\usepackage{ulem}
\usepackage[utf8]{inputenc}
\usepackage[russian]{babel}
\usepackage{minted}

\usepackage{color} %% это для отображения цвета в коде
\usepackage{listings} %% собственно, это и есть пакет listings

\lstset{ %
language=C++,                 % выбор языка для подсветки (здесь это С++)
basicstyle=\small\sffamily, % размер и начертание шрифта для подсветки кода
numbers=left,               % где поставить нумерацию строк (слева\справа)
%numberstyle=\tiny,           % размер шрифта для номеров строк
stepnumber=1,                   % размер шага между двумя номерами строк
numbersep=5pt,                % как далеко отстоят номера строк от подсвечиваемого кода
backgroundcolor=\color{white}, % цвет фона подсветки - используем \usepackage{color}
showspaces=false,            % показывать или нет пробелы специальными отступами
showstringspaces=false,      % показывать или нет пробелы в строках
showtabs=false,             % показывать или нет табуляцию в строках
frame=single,              % рисовать рамку вокруг кода
tabsize=2,                 % размер табуляции по умолчанию равен 2 пробелам
captionpos=t,              % позиция заголовка вверху [t] или внизу [b] 
breaklines=true,           % автоматически переносить строки (да\нет)
breakatwhitespace=false, % переносить строки только если есть пробел
escapeinside={\%*}{*)}   % если нужно добавить комментарии в коде
}


\begin{document}
\begin{titlepage}
\begin{center}
\textbf{МИНИСТЕРСТВО ОБРАЗОВАНИЯ И НАУКИ РОССИЙСКОЙ ФЕДЕРАЦИИ
\medskip
МОСКОВСКИЙ АВИАЦИОННЫЙ ИНСТИТУТ
(НАЦИОНАЛЬНЫЙ ИССЛЕДОВАТЕЛЬСКИЙ УНИВЕРСТИТЕТ)
\vfill\vfill
{\Huge ЛАБОРАТОРНАЯ РАБОТА №2} \\
по курсу объектно-ориентированное программирование
I семестр, 2021/22 уч. год}
\end{center}
\vfill

Студент \uline{\it {Каширин Кирилл Дмитриевич, группа М8О-208Б-20}\hfill}

Преподаватель \uline{\it {Дорохов Евгений Павлович}\hfill}

\vfill
\end{titlepage}

\subsection*{Условие}
Создать класс BitString для работы с 128-битовыми строками. Битовая строка
должна быть представлена двумя полями типа unsigned long long. Должны быть
реализованы все традиционные операции для работы с битами: and, or, xor, not.
Реализовать сдвиг влево shiftLeft и сдвиг вправо shiftRight на заданное количество битов. Реализовать операцию вычисления количества единичных битов, операции сравнения по количеству единичных битов. Реализовать операцию проверки
включения.
Исходный код лежит в 3 файлах:
\begin{enumerate}
\item main.cpp: основная программа, взаимодействие с пользователем посредством команд из меню
\item BitString.h:    описание класса адресов
\item BitString.cpp:  реализация класса адреса

\end{enumerate}
\pagebreak
\subsection*{Протокол работы}
BitString a: \\
000000000000000000000000000000000000000000000000 \\
010110111000000100000000000000000000000000000000 \\
00000000000000000111111010011101 \\
BitString b: \\
000000000000000000000000000000000000000000001010 \\
010110110001111000000000000000000000000000000000 \\
00000000000001010100010100000010 \\
AND \\
0000000000000000000000000000000000000000000000000 \\
1011011000000000000000000000000000000000000000000 \\
000000000000000100010000000000 \\
OR \\
0000000000000000000000000000000000000000000010100 \\
1011011100111110000000000000000000000000000000000 \\
000000000001010111111110011111 \\
XOR \\
0000000000000000000000000000000000000000000010100 \\
0000000100111110000000000000000000000000000000000 \\
000000000001010011101110011111 \\
NOT \\
1111111111111111111111111111111111111111111111111 \\
0100100011111101111111111111111111111111111111111 \\
111111111111111000000101100010 \\
1 in a \\
18 \\
0 \\
0 \\
Example with literal:00000000000000000000000000000 \\
00000000000001000111011110101011011000000000000000 \\
0000000000000000000000000000001100111011001100011 \\
Example with literal:00000000000000000000000000000 \\
00000000000001000111011110101011011000000000000000 \\
0000000000000000000000000000001100111011001100011 \\
\subsection*{Дневник отладки}
Проблем и ошибок при написании данной работы не возникло.

\subsection*{Недочёты}


\subsection*{Выводы}
В процессе выполнения работы я на практике познакомился с пользовательскими литералами. Как оказалось, у них есть свои преимущества. Они очень удобны и практичны. Использование этого средства позволяет получать из заданных типов данных какие то данные с использованием специального оператора.



\vfill
\pagebreak
\subsection*{Исходный код:}

{\Huge adress.h}
\inputminted
    {C++}{adress.h}
    
{\Huge adress.cpp}
\inputminted
    {C++}{adress.cpp}
    
{\Huge main.cpp}
\inputminted
    {C++}{main.cpp}
    
\end{document}